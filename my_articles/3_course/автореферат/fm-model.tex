\documentclass[12pt]{article}

\usepackage[utf8]{inputenc}
\usepackage[T2A]{fontenc}
\usepackage[russian,english]{babel}

\usepackage{amsmath}
\usepackage{amssymb}
\usepackage{geometry}
\geometry{a4paper,margin=2cm}

\newcommand{\sumphases}{\sum_{\alpha \in \{g,w,h\}}}

\begin{document}

\selectlanguage{russian}

\section*{Физико-математическая модель фильтрации с учетом образования газовых гидратов}

\subsection*{Насыщения и пористость}

\begin{equation}
  S_g + S_w + S_h = 1,
  \qquad 0 \le S_\alpha \le 1,
\end{equation}
где \(S_g, S_w, S_h\)~- насыщения газа, воды и гидрата соответственно.

Пористость \(\phi\) считаем зависящей от образования гидрата:
\begin{equation}\label{eq:phi_Sh}
  \phi(t,\mathbf{x}) = \phi_0(\mathbf{x})\,
  \bigl(1 - \beta_h S_h(t,\mathbf{x})\bigr),
\end{equation}
где \(\phi_0(\mathbf{x})\)~- начальная пористость коллекторной породы,
\(\beta_h \in [0,1]\)~- коэффициент, характеризующий,
какую долю порового объёма блокирует единица насыщения гидратом.

\subsection*{Абсолютная проницаемость как функция пористости}

Абсолютная проницаемость коллектора, связанная с геометрией пор,
описывается как функция пористости. Для простоты используем
степенную аппроксимацию типа Козени-Кармена:
\begin{equation}\label{eq:kabs_phi}
  k_{\mathrm{abs}}(\phi)
  = k_0\left(\frac{\phi}{\phi_0}\right)^{m},
\end{equation}
где \(k_0\)~- начальная абсолютная проницаемость при \(\phi = \phi_0\),
\(m \ge 1\)~- эмпирический показатель.

Подставляя \eqref{eq:phi_Sh} в \eqref{eq:kabs_phi}, получаем
явную зависимость абсолютной проницаемости от насыщения гидратом:
\begin{equation}\label{eq:kabs_Sh}
  k_{\mathrm{abs}}(S_h)
  = k_0\bigl(1 - \beta_h S_h\bigr)^{m}.
\end{equation}

Здесь \(k_{\mathrm{abs}}\) описывает \emph{только} изменение проводимости
скелета пористой среды из-за уменьшения эффективного порового объёма
при образовании твёрдого гидрата.

\subsection*{Закон сохранение массы (неразрывность)}

Запишем уравнение неразрывности для каждой моделируемой фазы (газ, вода, газовый гидрат).
При этом учтем, что в рамках рассматриваемой задачи обратный переход не рассматривается (моделирование до остановки). тогда получим, что:

\begin{subequations}\label{eq:mass-balance}
\begin{align}
  \frac{\partial}{\partial t}\bigl(\phi S_g \rho_g\bigr)
  + \nabla\cdot\bigl(\rho_g \mathbf{u}_g\bigr)
  &= - \nu_g R_h,
  \label{eq:mass-g}\\[0.3em]
  \frac{\partial}{\partial t}\bigl(\phi S_w \rho_w\bigr)
  + \nabla\cdot\bigl(\rho_w \mathbf{u}_w\bigr)
  &= - \nu_w R_h,
  \label{eq:mass-w}\\[0.3em]
  \frac{\partial}{\partial t}\bigl(\phi S_h \rho_h\bigr)
  &= + \nu_h R_h.
  \label{eq:mass-h}
\end{align}
\end{subequations}

Где \(\rho_\alpha = \rho_\alpha(p_\alpha,T)\)~- плотности фаз,
\(\mathbf{u}_\alpha\)~- фильтрационные скорости,
\(R_h\)~- скорость образования гидрата (масса на единицу объёма и времени),
\(\nu_g,\nu_w,\nu_h\)~- стехиометрические коэффициенты переноса массы.

\subsection*{Закон Дарси}

\subsubsection*{1. закон Дарси для однофазной фильтрации}

Для \emph{одной} подвижной фазы классический закон Дарси имеет следующий вид:
\begin{equation}\label{eq:darcy-abs}
  \mathbf{u}_\alpha
  = -\frac{k_{\mathrm{abs}}(\phi)}{\mu_\alpha}
    \Bigl(\nabla p_\alpha - \rho_\alpha \mathbf{g}\Bigr),
\end{equation}
где
\begin{itemize}
  \item \(k_{\mathrm{abs}}(\phi)\)~- абсолютная проницаемость,
        описывающая только свойства скелета порового пространства, зависит от свободной пористости;
  \item \(\mu_\alpha\)~- динамическая вязкость фазы \(\alpha\);
  \item \(p_\alpha\)~- давление фазы \(\alpha\);
  \item \(\mathbf{g}\)~- вектор ускорения силы тяжести.
\end{itemize}

Чтобы учитывать образование гидрата, представим \(k_{\mathrm{abs}}(\phi)\)  как функцию от насыщения 
\(k_{\mathrm{abs}}(S_h)\) из \eqref{eq:kabs_Sh}, таким образов учитываем что
поровое пространство становится менее проводящим по мере роста \(S_h\).

\subsubsection*{2. закон Дарси для многофазной фильтрации}

Для фильтрации в многофазном случае (газ + вода) необходимо учитывать \emph{относительные} фазовые
проницаемости \(k_{r\alpha}\), которые описывают дополнительное гидродинамического сопротивления фаз за счет конкуренции за один и тот же поровый объём, в котором происходит движение флюида.

Определим \emph{эффективную фазовую проницаемость} для фазы \(\alpha\):
\begin{equation}\label{eq:keff-def}
  k_{\mathrm{eff},\alpha}(S_g,S_w,S_h)
  := k_{\mathrm{abs}}(S_h)\,k_{r\alpha}(S_g,S_w,S_h),
\end{equation}
и \emph{по определению} относительную проницаемость как
\begin{equation}\label{eq:kr-def}
  k_{r\alpha}(S_g,S_w,S_h)
  = \frac{k_{\mathrm{eff},\alpha}}{k_{\mathrm{abs}}(S_h)}.
\end{equation}

То есть:
\begin{itemize}
  \item \(k_{\mathrm{abs}}(S_h)\)~- будем использовать абсолютную фазовую проницаемость определенную как \eqref{eq:kabs_Sh};
  \item \(k_{r\alpha}\)~- относительная фазовая проницаемость;
  \item \(k_{\mathrm{eff},\alpha}\)~- эффективная фазовая проницаемость\(\alpha\)
        в законе Дарси.
\end{itemize}

В \eqref{eq:keff-def} произведение \(k_{\mathrm{abs}} \cdot k_{r\alpha}\) есть эффективная фазовая проницаемость.
Тогда закон Дарси для каждой подвижной фазы \(\alpha \in \{g,w\}\)
\emph{только} через эффективную фазовую проницаемость с учетом образования гидрата запишется как:
\begin{equation}\label{eq:darcy-keff}
  \mathbf{u}_\alpha
  = -\frac{k_{\mathrm{eff},\alpha}(S_g,S_w,S_h)}{\mu_\alpha}
    \Bigl(\nabla p_\alpha - \rho_\alpha \mathbf{g}\Bigr),
  \qquad \alpha \in \{g,w\}.
\end{equation}

Где связь с абсолютной проницаемостью и насыщениями задаётся
через \eqref{eq:kabs_Sh} и \eqref{eq:keff-def}:
\[
  k_{\mathrm{eff},\alpha}(S_g,S_w,S_h)
  = k_0\bigl(1 - \beta_h S_h\bigr)^m
    \,k_{r\alpha}(S_g,S_w,S_h).
\]

\section*{Фазовый переход газа и вода $\rightarrow$ газовый гидрат}

В этом разделе будет показано, как фазовый переход учитываетс в законах сохранения массы и энергии.

\subsection*{Локальное уравнение баланса массы для фазы}

Рассмотрим произвольную фазу \(\alpha \in \{g,w,h\}\) (газ, вода, гидрат).
Запишем локальное уравнение баланса массы для этой фазы в пористом среде:
\begin{equation}\label{eq:local-mass-alpha}
  \frac{\partial}{\partial t}\bigl(\phi S_\alpha \rho_\alpha\bigr)
  + \nabla\cdot\bigl(\rho_\alpha \mathbf{u}_\alpha\bigr)
  = \Gamma_\alpha,
\end{equation}
где
\begin{itemize}
  \item \(\phi\)~--- пористость;
  \item \(S_\alpha\)~--- насыщение фазы \(\alpha\) (доля порового объёма);
  \item \(\rho_\alpha\)~--- плотность фазы \(\alpha\);
  \item \(\mathbf{u}_\alpha\)~--- фильтрационная скорость фазы \(\alpha\)
        (скорость Дарси);
  \item \(\Gamma_\alpha\)~--- источник (если \(>0\)) или сток (если \(<0\)) массы фазы
        за счёт фазового перехода.
\end{itemize}

\subsection*{Определение скорости фазового перехода $R_h$}

Введём скалярную величину \(R_h(t,\mathbf{x})\):
\begin{itemize}
  \item \(R_h\)~--- скорость образования гидрата \emph{в пересчёте на массу гидрата} на единицу порового объема и времени.
\end{itemize}

Тогда:
\begin{equation}
  \Gamma_h = \nu_h R_h,
\end{equation}
то есть правая часть уравнения для гидратной фазы может быть записана как
\begin{equation}\label{eq:mass-h-derivation}
  \frac{\partial}{\partial t}\bigl(\phi S_h \rho_h\bigr)
  = \Gamma_h
  = \nu_h R_h.
\end{equation}

Для газа и воды \(R_h\) даёт \emph{отрицательные} источники (стоки):
\begin{equation}
  \Gamma_g = - \nu_g R_h,
  \qquad
  \Gamma_w = - \nu_w R_h,
\end{equation}
то есть
\begin{subequations}\label{eq:mass-gw-derivation}
\begin{align}
  \frac{\partial}{\partial t}\bigl(\phi S_g \rho_g\bigr)
  + \nabla\cdot\bigl(\rho_g \mathbf{u}_g\bigr)
  &= - \nu_g R_h,\\[0.3em]
  \frac{\partial}{\partial t}\bigl(\phi S_w \rho_w\bigr)
  + \nabla\cdot\bigl(\rho_w \mathbf{u}_w\bigr)
  &= - \nu_w R_h.
\end{align}
\end{subequations}

Коэффициенты \(\nu_g\), \(\nu_w\), \(\nu_h\) здесь:
\begin{itemize}
  \item могут быть выражены через стехиометрию реакции и молярные массы,
  \item можно представить как эмпирические \emph{эффективные коэффициенты}, которые учитывают, сколько массы газа и воды уходит на единицу массы образующегося гидрата.
\end{itemize}

За счет \eqref{eq:mass-h-derivation} и \eqref{eq:mass-gw-derivation}
фазовый переход можно \emph{явно} учесть в уравнениях неразрывности:
масса газа и воды убывает, масса гидрата возрастает.

\subsection*{Связь скорости фазового перехода $R_h$ с насыщением гидратом $S_h$}

Если предположить, что плотность гидрата \(\rho_h\) и пористость \(\phi\)
меняются медленно (или считаем их константами на первом приближении),
из \eqref{eq:mass-h-derivation} получаем:
\begin{equation}\label{eq:Sh-from-Rh}
  \rho_h \phi \,\frac{\partial S_h}{\partial t}
  = \nu_h R_h
  \quad\Longrightarrow\quad
  \frac{\partial S_h}{\partial t}
  = \frac{\nu_h}{\rho_h \phi} R_h.
\end{equation}

Таким образом, используя функцию скорости образования гидрата \(R_h\)
(кинетику образования гидрата), можно определять насыщение порового пространства газовым гидратом \(S_h(t,\mathbf{x})\).

\subsection*{Закон сохранения энергии}

Учитывая что рассматриваемые процесс образования гидратов в аномальных пластах зависит не только от давления но и температуры, необходимо так же учитывать закон сохранения энергии. 

\begin{equation}\label{eq:energy}
\begin{aligned}
  \frac{\partial}{\partial t}
  \Bigl[
    (1-\phi)\,\rho_r c_r T
    + \phi \sumphases S_\alpha \rho_\alpha c_\alpha T
  \Bigr]
  &+
  \nabla\cdot\Bigl(
    -\lambda_{\mathrm{eff}}\nabla T
    + \sum_{\alpha\in\{g,w\}} \rho_\alpha h_\alpha \mathbf{u}_\alpha
  \Bigr)
  = - L_h R_h .
\end{aligned}
\end{equation}

Здесь \(\rho_r, c_r\)~- плотность и теплоёмкость матрицы,
\(c_\alpha\)~- теплоёмкости фаз,
\(\lambda_{\mathrm{eff}}\)~- эффективная теплопроводность пористой среды,
\(h_\alpha\)~- удельные энтальпии фаз,
\(L_h\)~- удельная скрытая теплота образования гидрата.

\(\displaystyle
\phi \sum_{\alpha\in\{g,w,h\}} S_\alpha \rho_\alpha c_\alpha T
\)
— внутренняя энергия фаз в порах.  
В каждой фазе присутствует теплоёмкость \(c_\alpha\),
плотность \(\rho_\alpha\) и температура \(T\).

\[
\nabla\cdot
\Bigl(
\sum_{\alpha\in\{g,w\}} \rho_\alpha h_\alpha \mathbf{u}_\alpha
\Bigr)
\] — конвективный перенос тепла движущимися фазами, где :
\begin{itemize}
  \item \(\rho_\alpha \mathbf{u}_\alpha\) — массовый поток фазы,
  \item \(h_\alpha\) — удельная энтальпия фазы,
\end{itemize}


\subsection*{Кинетика образования гидрата}
В соответствии с моделью И.Ленгмюра, образование газового гидрата ограничено предельной равновесной концентрацией, определяемой как функция от давления и температуры, а именно:  
\begin{subequations}\label{eq:hydrate-kinetics}
\begin{align}
  S_h^{\mathrm{eq}}(p_g,T)
  &= S_{h,\max}\,
     \frac{p_g}{p_g + P_L(T)},
  \label{eq:Sh-eq-langmuir}\\[0.4em]
  R_h
  &= \rho_h\,k_L \Bigl(S_h^{\mathrm{eq}}(p_g,T) - S_h\Bigr),
  \label{eq:Rh-langmuir}
\end{align}
\end{subequations}

где \(S_{h,\max}\)~- максимальное гидратное насыщение,
\(P_L(T)\)~- параметр Ленгмюра (характерное давление насыщения),
\(k_L\)~- кинетический коэффициент.

Из баланса массы гидрата \eqref{eq:mass-h} при медленном изменении \(\rho_h\)
получаем:
\begin{equation}\label{eq:Sh-evol}
  \rho_h \phi \frac{\partial S_h}{\partial t} = \nu_h R_h
  \quad\Longrightarrow\quad
  \frac{\partial S_h}{\partial t}
  = \frac{\nu_h}{\rho_h \phi} R_h.
\end{equation}

Подставляя \eqref{eq:Rh-langmuir}, имеем:
\begin{equation}\label{eq:Sh-evol-langmuir}
  \frac{\partial S_h}{\partial t}
  = \frac{\nu_h k_L}{\phi}
    \Bigl(S_h^{\mathrm{eq}}(p_g,T) - S_h\Bigr).
\end{equation}

Решив уравнение \eqref{eq:Sh-evol-langmuir} относительно насыщения гидратом и подставив его в \eqref{eq:phi_Sh} и \eqref{eq:kabs_phi}, можно написать:
\begin{equation}\label{eq:kabs-static}
  k_{\mathrm{abs}}(S_h)
  = k_0 \left(1 - \beta_h S_h\right)^{m}.
\end{equation}

Получив квазистационарное уравнение, учитывающее влияние роста гидратного насыщения на абсолютную проницаемость.

\subsection*{Дополнительная динамика кольматации (уравнение Леонтьева)}

Чтобы учесть перенос и накопление частиц гидрата (суффозия, кольматация) на проницаемость призабойной зоны скважины, введём
безразмерный множитель кольматации \(F_{\mathrm{colm}}(t,\mathbf{x})\) 
и определим эффективную фазовую проницаемость как
\begin{equation}\label{eq:keff-full}
  k_{\mathrm{eff},\alpha}(S_g,S_w,S_h,t)
  = k_{\mathrm{abs}}(S_h)\,
    F_{\mathrm{colm}}(t,\mathbf{x})\,
    k_{r\alpha}(S_g,S_w,S_h).
\end{equation}

Для \(F_{\mathrm{colm}}\) запишем уравнение Леонтьева:
\begin{subequations}\label{eq:leontiev}
\begin{align}
  \frac{\partial F_{\mathrm{colm}}}{\partial t}
  &= - \alpha \, \bigl|\mathbf{u}_\mathrm{f}\bigr|\,
      S_h\, F_{\mathrm{colm}},
      \label{eq:leontiev-F}\\[0.4em]
  \mathbf{u}_\mathrm{f}
  &= \omega_g \mathbf{u}_g + \omega_w \mathbf{u}_w,
\end{align}
\end{subequations}
где \(\alpha\)~- коэффициент кольматации,
\(\mathbf{u}_\mathrm{f}\)~- характерная скорость фильтрации смеси,
\(\omega_g,\omega_w\)~- весовые коэффициенты (вклад газа и воды).

Решение \eqref{eq:leontiev-F} при известном \(S_h\) и \(\mathbf{u}_\mathrm{f}\):
\begin{equation}\label{eq:leontiev-sol}
  F_{\mathrm{colm}}(t,\mathbf{x})
  = \exp\Bigl(
      -\alpha \int_0^t
        \bigl|\mathbf{u}_\mathrm{f}(\tau,\mathbf{x})\bigr|\,
        S_h(\tau,\mathbf{x})\, d\tau
    \Bigr),
\end{equation}
если \(F_{\mathrm{colm}}(0,\mathbf{x}) = 1\).

Тогда \(k_{\mathrm{eff},\alpha}\) из \eqref{eq:keff-full}
учитывает три эффекта:
\begin{itemize}
  \item изменение свободной пористой из-за гидрата (\(k_{\mathrm{abs}}(S_h)\));
  \item многофазность (\(k_{r\alpha}\));
  \item динамическую кольматацию \(F_{\mathrm{colm}}\) каналов.
\end{itemize}

\subsection*{Возможности предлагаемой физико-математической модели}

Таким образом, предлагаемая модель включает в себя:
\begin{itemize}
  \item массовые балансы фаз \eqref{eq:mass-balance};
  \item закон Дарси в виде \eqref{eq:darcy-keff} с
        эффективной фазовой проницаемостью \eqref{eq:keff-full};
  \item связь пористости с гидратным насыщением \eqref{eq:phi_Sh}, как следствие динамическое изменение проницаемости коллектора ;
  \item зависимость абсолютной проницаемости от пористости, насыщения газовым гидратом;
  \item кинетику образования гидрата по Ленгмюру \eqref{eq:hydrate-kinetics},
        эволюцию \(S_h\) \eqref{eq:Sh-evol-langmuir};
  \item баланс энергии \eqref{eq:energy} с источником-стоком
        скрытой теплоты фазового перехода \(-L_h R_h\);
  \item дополнительную динамику кольматации по Леонтьеву
        через \(F_{\mathrm{colm}}\) \eqref{eq:leontiev}, \eqref{eq:leontiev-sol}.
\end{itemize}

Данная модель позволит:
\begin{itemize}
  \item \emph{учитывать зависимость абсолютной} проницаемости \(k_{\mathrm{abs}}\) от насыщением порового пространства газовым гидратом;
  \item \emph{учитывать изменение относительные фазовой} проницаемости \(k_{r\alpha}\);
  \item Определяемые в соответствии с моделью \(k_{\mathrm{abs}}, F_{\mathrm{colm}}, k_{r\alpha}\) должны позволить определять динамику работы скважины в условиях аномально низких пластовых давлений и температур, прогнозировать вынужденные остановки для ГТМ, подбирать оптимальные режимы работы.
\end{itemize}

\end{document}
