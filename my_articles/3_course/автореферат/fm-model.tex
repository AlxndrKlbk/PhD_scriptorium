\documentclass[12pt]{article}

\usepackage[utf8]{inputenc}
\usepackage[T2A]{fontenc}
\usepackage[russian,english]{babel}

\usepackage{amsmath}
\usepackage{amssymb}
\usepackage{geometry}
\geometry{a4paper,margin=2cm}

\newcommand{\sumphases}{\sum_{\alpha \in \{g,w,h\}}}

\begin{document}

\selectlanguage{russian}

\section*{Физико-математическая модель фильтрации с учетом образования газовых гидратов}

\subsection*{Насыщения и пористость}

\begin{equation}
  S_g + S_w + S_h = 1,
  \qquad 0 \le S_\alpha \le 1,
\end{equation}
где \(S_g, S_w, S_h\)~--- насыщения газа, воды и гидрата соответственно.

Пористость \(\phi\) считаем зависящей от образования гидрата:
\begin{equation}\label{eq:phi_Sh}
  \phi(t,\mathbf{x}) = \phi_0(\mathbf{x})\,
  \bigl(1 - \beta_h S_h(t,\mathbf{x})\bigr),
\end{equation}
где \(\phi_0(\mathbf{x})\)~--- начальная пористость коллекторной породы,
\(\beta_h \in [0,1]\)~--- коэффициент, характеризующий,
какую долю порового объёма блокирует единица насыщения гидратом.

\subsection*{Абсолютная проницаемость как функция пористости}

Абсолютная проницаемость коллектора, связанная с геометрией пор,
описывается как функция пористости. Для простоты используем
степенную аппроксимацию типа Козени--Кармена:
\begin{equation}\label{eq:kabs_phi}
  k_{\mathrm{abs}}(\phi)
  = k_0\left(\frac{\phi}{\phi_0}\right)^{m},
\end{equation}
где \(k_0\)~--- начальная абсолютная проницаемость при \(\phi = \phi_0\),
\(m \ge 1\)~--- эмпирический показатель.

Подставляя \eqref{eq:phi_Sh} в \eqref{eq:kabs_phi}, получаем
явную зависимость абсолютной проницаемости от насыщения гидратом:
\begin{equation}\label{eq:kabs_Sh}
  k_{\mathrm{abs}}(S_h)
  = k_0\bigl(1 - \beta_h S_h\bigr)^{m}.
\end{equation}

Здесь \(k_{\mathrm{abs}}\) описывает \emph{только} изменение проводимости
скелета пористой среды из-за уменьшения эффективного порового объёма
при образовании твёрдого гидрата.

\subsection*{Закон сохранения массы (неразрывность)}

Запишем уравнение неразрывности для каждой моделируемой фазы (газ, вода, газовый гидрат).
При этом учтём, что в рамках рассматриваемой задачи обратный переход не рассматривается
(моделирование до остановки). Тогда получим:
\begin{subequations}\label{eq:mass-balance}
\begin{align}
  \frac{\partial}{\partial t}\bigl(\phi S_g \rho_g\bigr)
  + \nabla\cdot\bigl(\rho_g \mathbf{u}_g\bigr)
  &= - \nu_g R_h,
  \label{eq:mass-g}\\[0.3em]
  \frac{\partial}{\partial t}\bigl(\phi S_w \rho_w\bigr)
  + \nabla\cdot\bigl(\rho_w \mathbf{u}_w\bigr)
  &= - \nu_w R_h,
  \label{eq:mass-w}\\[0.3em]
  \frac{\partial}{\partial t}\bigl(\phi S_h \rho_h\bigr)
  &= + \nu_h R_h.
  \label{eq:mass-h}
\end{align}
\end{subequations}

Здесь \(\rho_\alpha = \rho_\alpha(p_\alpha,T)\)~--- плотности фаз,
\(\mathbf{u}_\alpha\)~--- фильтрационные скорости,
\(R_h\)~--- скорость образования гидрата (масса на единицу объёма и времени),
\(\nu_g,\nu_w,\nu_h\)~--- стехиометрические коэффициенты переноса массы.

\subsection*{Закон Дарси}

\subsubsection*{1. Закон Дарси для однофазной фильтрации}

Для \emph{одной} подвижной фазы классический закон Дарси имеет вид:
\begin{equation}\label{eq:darcy-abs}
  \mathbf{u}_\alpha
  = -\frac{k_{\mathrm{abs}}(\phi)}{\mu_\alpha}
    \bigl(\nabla p_\alpha\bigr),
\end{equation}
где
\begin{itemize}
  \item \(k_{\mathrm{abs}}(\phi)\)~--- абсолютная проницаемость,
        описывающая только свойства скелета порового пространства, зависит от свободной пористости;
  \item \(\mu_\alpha\)~--- динамическая вязкость фазы \(\alpha\);
  \item \(p_\alpha\)~--- давление фазы \(\alpha\);
  \item \(\mathbf{g}\)~--- вектор ускорения силы тяжести.
\end{itemize}

Чтобы учитывать образование гидрата, представим \(k_{\mathrm{abs}}(\phi)\)
как функцию от насыщения \(k_{\mathrm{abs}}(S_h)\) из \eqref{eq:kabs_Sh},
таким образом учитывая, что поровое пространство становится менее проводящим
по мере роста \(S_h\).

\subsubsection*{2. Закон Дарси для многофазной фильтрации}

Для фильтрации в многофазном случае (газ + вода) необходимо учитывать
\emph{относительные} фазовые проницаемости \(k_{r\alpha}\),
которые описывают дополнительное гидродинамическое сопротивление фаз
за счёт конкуренции за один и тот же поровый объём, в котором происходит движение флюида.

Определим \emph{эффективную фазовую проницаемость} для фазы \(\alpha\):
\begin{equation}\label{eq:keff-def}
  k_{\mathrm{eff},\alpha}(S_g,S_w,S_h)
  := k_{\mathrm{abs}}(S_h)\,k_{r\alpha}(S_g,S_w,S_h),
\end{equation}
и \emph{по определению} относительную проницаемость как
\begin{equation}\label{eq:kr-def}
  k_{r\alpha}(S_g,S_w,S_h)
  = \frac{k_{\mathrm{eff},\alpha}}{k_{\mathrm{abs}}(S_h)}.
\end{equation}

То есть:
\begin{itemize}
  \item \(k_{\mathrm{abs}}(S_h)\)~--- абсолютная фазовая проницаемость,
        определённая как \eqref{eq:kabs_Sh};
  \item \(k_{r\alpha}\)~--- относительная фазовая проницаемость;
  \item \(k_{\mathrm{eff},\alpha}\)~--- эффективная фазовая проницаемость
        фазы \(\alpha\) в законе Дарси.
\end{itemize}

В \eqref{eq:keff-def} произведение \(k_{\mathrm{abs}} \cdot k_{r\alpha}\)
есть эффективная фазовая проницаемость.
Тогда закон Дарси для каждой подвижной фазы \(\alpha \in \{g,w\}\)
\emph{только} через эффективную фазовую проницаемость с учетом образования гидрата
запишется как:
\begin{equation}\label{eq:darcy-keff}
  \mathbf{u}_\alpha
  = -\frac{k_{\mathrm{eff},\alpha}(S_g,S_w,S_h)}{\mu_\alpha}
    \bigl(\nabla p_\alpha\bigr),
  \qquad \alpha \in \{g,w\}.
\end{equation}

Связь с абсолютной проницаемостью и насыщениями задаётся
через \eqref{eq:kabs_Sh} и \eqref{eq:keff-def}:
\[
  k_{\mathrm{eff},\alpha}(S_g,S_w,S_h)
  = k_0\bigl(1 - \beta_h S_h\bigr)^m
    \,k_{r\alpha}(S_g,S_w,S_h).
\]

\section*{Фазовый переход газа и воды в газовый гидрат}

В этом разделе будет показано, как фазовый переход учитывается
в законах сохранения массы и энергии.

\subsection*{Локальное уравнение баланса массы для фазы}

Рассмотрим произвольную фазу \(\alpha \in \{g,w,h\}\) (газ, вода, гидрат).
Запишем локальное уравнение баланса массы для этой фазы в пористой среде:
\begin{equation}\label{eq:local-mass-alpha}
  \frac{\partial}{\partial t}\bigl(\phi S_\alpha \rho_\alpha\bigr)
  + \nabla\cdot\bigl(\rho_\alpha \mathbf{u}_\alpha\bigr)
  = \Gamma_\alpha,
\end{equation}
где
\begin{itemize}
  \item \(\phi\)~--- пористость;
  \item \(S_\alpha\)~--- насыщение фазы \(\alpha\) (доля порового объёма);
  \item \(\rho_\alpha\)~--- плотность фазы \(\alpha\);
  \item \(\mathbf{u}_\alpha\)~--- фильтрационная скорость фазы \(\alpha\)
        (скорость Дарси);
  \item \(\Gamma_\alpha\)~--- источник (если \(>0\)) или сток (если \(<0\)) массы фазы
        за счёт фазового перехода.
\end{itemize}

\subsection*{Определение скорости фазового перехода $R_h$}

Введём скалярную величину \(R_h(t,\mathbf{x})\):
\begin{itemize}
  \item \(R_h\)~--- скорость образования гидрата \emph{в пересчёте на массу гидрата}
        на единицу порового объема и времени.
\end{itemize}

Тогда:
\begin{equation}
  \Gamma_h = \nu_h R_h,
\end{equation}
то есть правая часть уравнения для гидратной фазы может быть записана как
\begin{equation}\label{eq:mass-h-derivation}
  \frac{\partial}{\partial t}\bigl(\phi S_h \rho_h\bigr)
  = \Gamma_h
  = \nu_h R_h.
\end{equation}

Для газа и воды \(R_h\) даёт \emph{отрицательные} источники (стоки):
\begin{equation}
  \Gamma_g = - \nu_g R_h,
  \qquad
  \Gamma_w = - \nu_w R_h,
\end{equation}
то есть
\begin{subequations}\label{eq:mass-gw-derivation}
\begin{align}
  \frac{\partial}{\partial t}\bigl(\phi S_g \rho_g\bigr)
  + \nabla\cdot\bigl(\rho_g \mathbf{u}_g\bigr)
  &= - \nu_g R_h,\\[0.3em]
  \frac{\partial}{\partial t}\bigl(\phi S_w \rho_w\bigr)
  + \nabla\cdot\bigl(\rho_w \mathbf{u}_w\bigr)
  &= - \nu_w R_h.
\end{align}
\end{subequations}

Коэффициенты \(\nu_g\), \(\nu_w\), \(\nu_h\) здесь:
\begin{itemize}
  \item могут быть выражены через стехиометрию реакции и молярные массы,
  \item можно представить как эмпирические \emph{эффективные коэффициенты},
        которые учитывают, сколько массы газа и воды уходит на единицу массы образующегося гидрата.
\end{itemize}

За счёт \eqref{eq:mass-h-derivation} и \eqref{eq:mass-gw-derivation}
фазовый переход можно \emph{явно} учесть в уравнениях неразрывности:
масса газа и воды убывает, масса гидрата возрастает.

\subsection*{Связь скорости фазового перехода $R_h$ с насыщением гидратом $S_h$}

Если предположить, что плотность гидрата \(\rho_h\) и пористость \(\phi\)
меняются медленно (или считать их константами на первом приближении),
из \eqref{eq:mass-h-derivation} получаем:
\begin{equation}\label{eq:Sh-from-Rh}
  \rho_h \phi \,\frac{\partial S_h}{\partial t}
  = \nu_h R_h
  \quad\Longrightarrow\quad
  \frac{\partial S_h}{\partial t}
  = \frac{\nu_h}{\rho_h \phi} R_h.
\end{equation}

Таким образом, используя функцию скорости образования гидрата \(R_h\)
(кинетику образования гидрата), можно определять насыщение порового пространства
газовым гидратом \(S_h(t,\mathbf{x})\).

\subsection*{Закон сохранения энергии}

Учитывая, что рассматриваемый процесс образования гидратов в аномальных пластах
зависит не только от давления, но и от температуры,
необходимо также учитывать закон сохранения энергии:
\begin{equation}\label{eq:energy}
\begin{aligned}
  \frac{\partial}{\partial t}
  \Bigl[
    (1-\phi)\,\rho_r c_r T
    + \phi \sumphases S_\alpha \rho_\alpha c_\alpha T
  \Bigr]
  &+
  \nabla\cdot\Bigl(
    -\lambda_{\mathrm{eff}}\nabla T
    + \sum_{\alpha\in\{g,w\}} \rho_\alpha h_\alpha \mathbf{u}_\alpha
  \Bigr)
  = - L_h R_h .
\end{aligned}
\end{equation}

Здесь \(\rho_r, c_r\)~--- плотность и теплоёмкость матрицы,
\(c_\alpha\)~--- теплоёмкости фаз,
\(\lambda_{\mathrm{eff}}\)~--- эффективная теплопроводность пористой среды,
\(h_\alpha\)~--- удельные энтальпии фаз,
\(L_h\)~--- удельная скрытая теплота образования гидрата.

Величина
\[
\phi \sum_{\alpha\in\{g,w,h\}} S_\alpha \rho_\alpha c_\alpha T
\]
описывает внутреннюю энергию фаз в порах.
В каждой фазе присутствует теплоёмкость \(c_\alpha\),
плотность \(\rho_\alpha\) и температура \(T\).

Член
\[
\nabla\cdot
\Bigl(
\sum_{\alpha\in\{g,w\}} \rho_\alpha h_\alpha \mathbf{u}_\alpha
\Bigr)
\]
описывает конвективный перенос тепла движущимися фазами, где
\begin{itemize}
  \item \(\rho_\alpha \mathbf{u}_\alpha\)~--- массовый поток фазы,
  \item \(h_\alpha\)~--- удельная энтальпия фазы.
\end{itemize}

\subsection*{Кинетика образования гидрата}

В соответствии с моделью Ленгмюра, образование газового гидрата ограничено
предельной равновесной концентрацией, определяемой как функция от давления и температуры:
\begin{subequations}\label{eq:hydrate-kinetics}
\begin{align}
  S_h^{\mathrm{eq}}(p_g,T)
  &= S_{h,\max}\,
     \frac{p_g}{p_g + P_L(T)},
  \label{eq:Sh-eq-langmuir}\\[0.4em]
  R_h
  &= \rho_h\,k_L \Bigl(S_h^{\mathrm{eq}}(p_g,T) - S_h\Bigr),
  \label{eq:Rh-langmuir}
\end{align}
\end{subequations}
где \(S_{h,\max}\)~--- максимальное гидратное насыщение,
\(P_L(T)\)~--- параметр Ленгмюра (характерное давление насыщения),
\(k_L\)~--- кинетический коэффициент.

Из баланса массы гидрата \eqref{eq:mass-h} при медленном изменении \(\rho_h\)
получаем:
\begin{equation}\label{eq:Sh-evol}
  \rho_h \phi \frac{\partial S_h}{\partial t} = \nu_h R_h
  \quad\Longrightarrow\quad
  \frac{\partial S_h}{\partial t}
  = \frac{\nu_h}{\rho_h \phi} R_h.
\end{equation}

Подставляя \eqref{eq:Rh-langmuir}, имеем:
\begin{equation}\label{eq:Sh-evol-langmuir}
  \frac{\partial S_h}{\partial t}
  = \frac{\nu_h k_L}{\phi}
    \Bigl(S_h^{\mathrm{eq}}(p_g,T) - S_h\Bigr).
\end{equation}

Решив уравнение \eqref{eq:Sh-evol-langmuir} относительно насыщения гидратом
и подставив его в \eqref{eq:phi_Sh} и \eqref{eq:kabs_phi}, можно написать:
\begin{equation}\label{eq:kabs-static}
  k_{\mathrm{abs}}(S_h)
  = k_0 \left(1 - \beta_h S_h\right)^{m}.
\end{equation}

Таким образом получаем квазистационарное уравнение,
учитывающее влияние роста гидратного насыщения на абсолютную проницаемость.

\subsection*{Динамика кольматации порового пространства}

Чтобы учесть перенос и накопление частиц гидрата (суффозия, кольматация)
на проницаемость призабойной зоны скважины, введём
безразмерный множитель кольматации \(F_{\mathrm{colm}}(t,\mathbf{x})\)
и определим эффективную фазовую проницаемость как
\begin{equation}\label{eq:keff-full}
  k_{\mathrm{eff},\alpha}(S_g,S_w,S_h,t)
  = k_{\mathrm{abs}}(S_h)\,
    F_{\mathrm{colm}}(t,\mathbf{x})\,
    k_{r\alpha}(S_g,S_w,S_h).
\end{equation}

Для \(F_{\mathrm{colm}}\) запишем уравнение Леонтьева:
\begin{subequations}\label{eq:leontiev}
\begin{align}
  \frac{\partial F_{\mathrm{colm}}}{\partial t}
  &= - \alpha \, \bigl|\mathbf{u}_\mathrm{f}\bigr|\,
      S_h\, F_{\mathrm{colm}},
      \label{eq:leontiev-F}\\[0.4em]
  \mathbf{u}_\mathrm{f}
  &= \omega_g \mathbf{u}_g + \omega_w \mathbf{u}_w,
\end{align}
\end{subequations}
где \(\alpha\)~--- коэффициент кольматации,
\(\mathbf{u}_\mathrm{f}\)~--- характерная скорость фильтрации смеси,
\(\omega_g,\omega_w\)~--- весовые коэффициенты (вклад газа и воды).

Решение \eqref{eq:leontiev-F} при известном \(S_h\) и \(\mathbf{u}_\mathrm{f}\):
\begin{equation}\label{eq:leontiev-sol}
  F_{\mathrm{colm}}(t,\mathbf{x})
  = \exp\Bigl(
      -\alpha \int_0^t
        \bigl|\mathbf{u}_\mathrm{f}(\tau,\mathbf{x})\bigr|\,
        S_h(\tau,\mathbf{x})\, d\tau
    \Bigr),
\end{equation}
если \(F_{\mathrm{colm}}(0,\mathbf{x}) = 1\).

Тогда \(k_{\mathrm{eff},\alpha}\) из \eqref{eq:keff-full}
учитывает три эффекта:
\begin{itemize}
  \item изменение свободной пористости из-за гидрата (\(k_{\mathrm{abs}}(S_h)\));
  \item многофазность (\(k_{r\alpha}\));
  \item динамическую кольматацию \(F_{\mathrm{colm}}\) каналов.
\end{itemize}

\section*{Численная схема для закона сохранения энергии}

Для численного решения уравнения сохранения энергии \eqref{eq:energy}
используем метод контрольных объёмов (finite volume method, FVM).
Пространственная область разбивается на непересекающееся семейство
контрольных объёмов
\[
  \{V_i\}_{i=1}^{N},
\]
каждый из которых имеет объём \(|V_i|\) и набор соседей \(N(i)\).
Через \(A_{ij}\) обозначим площадь общей грани между объёмами \(V_i\) и \(V_j\),
через \(\mathbf{n}_{ij}\)~--- внешнюю нормаль, направленную из \(V_i\) в \(V_j\),
через \(d_{ij}\)~--- расстояние между центрами ячеек \(i\) и \(j\).

\subsection*{Интегральная форма уравнения энергии}

Интегрируя \eqref{eq:energy} по контрольному объёму \(V_i\), получаем
\begin{equation}\label{eq:energy-integral}
  \frac{\partial}{\partial t}
  \int_{V_i}
  \left[
    (1-\phi)\,\rho_r c_r T
    +
    \phi \sum_{\alpha\in\{g,w,h\}}
      S_\alpha \rho_\alpha c_\alpha T
  \right] dV
  +
  \oint_{\partial V_i}
  \left(
    -\lambda_{\mathrm{eff}} \nabla T
    +
    \sum_{\alpha\in\{g,w\}}
      \rho_\alpha h_\alpha \mathbf{u}_\alpha
  \right)\!\cdot \mathbf{n}\, dA
  =
  -\int_{V_i} L_h R_h\, dV .
\end{equation}

Введём обозначение для внутренней энергии в ячейке \(V_i\):
\begin{equation}\label{eq:E-i-def}
  E_i
  =
  (1-\phi_i)\,\rho_{r,i} c_{r,i} T_i
  +
  \phi_i
  \sum_{\alpha\in\{g,w,h\}}
    S_{\alpha,i} \rho_{\alpha,i} c_{\alpha,i} T_i .
\end{equation}

Тогда интегральное уравнение \eqref{eq:energy-integral} в дискретной форме
(для шага времени \(\Delta t = t^{n+1}-t^{n}\)) запишется как
\begin{equation}\label{eq:energy-fvm}
  \frac{|V_i|}{\Delta t}
  \left(
    E_i^{n+1} - E_i^{n}
  \right)
  +
  \sum_{j\in N(i)}
  \left(
    q_{ij}^{\mathrm{cond},\,n+1}
    +
    q_{ij}^{\mathrm{conv},\,n+1}
  \right)
  =
  - |V_i|\, L_h R_{h,i}^{n+1}.
\end{equation}

Здесь:
\begin{itemize}
  \item \(E_i^{n}, E_i^{n+1}\)~--- внутренняя энергия в ячейке \(V_i\)
        на шагах \(t^n\) и \(t^{n+1}\), определяемая по \eqref{eq:E-i-def};
  \item \(q_{ij}^{\mathrm{cond},\,n+1}\)~--- теплопроводный поток
        через грань между ячейками \(i\) и \(j\) на шаге \(t^{n+1}\);
  \item \(q_{ij}^{\mathrm{conv},\,n+1}\)~--- конвективный поток энтальпии
        через ту же грань;
  \item \(R_{h,i}^{n+1}\)~--- скорость образования гидрата в объёме \(V_i\)
        на шаге \(t^{n+1}\), определяемая по кинетике Ленгмюра
        \eqref{eq:hydrate-kinetics};
  \item правая часть \(-|V_i|\,L_h R_{h,i}^{n+1}\) описывает поглощение
        теплоты при фазовом переходе (образовании гидрата).
\end{itemize}

\subsection*{Дискретизация теплопроводного потока}

Теплопроводный вклад в плотность теплового потока задаётся выражением
\[
  \mathbf{q}^{\mathrm{cond}}
  =
  -\lambda_{\mathrm{eff}} \nabla T .
\]

Поток через грань между ячейками \(V_i\) и \(V_j\) дискретизуем
по нормальному направлению:
\begin{equation}\label{eq:q-cond}
  q_{ij}^{\mathrm{cond},\,n+1}
  =
  -\lambda_{\mathrm{eff},ij}^{\,n+1}
  \left(
    \frac{T_j^{n+1} - T_i^{n+1}}{d_{ij}}
  \right)
  A_{ij}.
\end{equation}

Здесь:
\begin{itemize}
  \item \(\lambda_{\mathrm{eff},ij}^{\,n+1}\)~--- эффективная теплопроводность
        на грани \(i\text{--}j\) (полученная, например, как гармоническое или
        арифметическое среднее значений в ячейках \(i\) и \(j\));
  \item \(T_i^{n+1}, T_j^{n+1}\)~--- температуры в центрах ячеек
        на шаге \(t^{n+1}\);
  \item \(d_{ij}\)~--- расстояние между центрами ячеек;
  \item \(A_{ij}\)~--- площадь грани между ячейками.
\end{itemize}

Отрицательный знак в \eqref{eq:q-cond} отражает направление потока
от более высокой температуры к более низкой.

\subsection*{Дискретизация конвективного переноса энтальпии}

Конвективная часть потока тепла связана с переносом энтальпии
движущимися фазами газа и воды:
\[
  \mathbf{q}^{\mathrm{conv}}
  =
  \sum_{\alpha\in\{g,w\}}
    \rho_\alpha h_\alpha \mathbf{u}_\alpha.
\]

Поток через грань между ячейками \(V_i\) и \(V_j\) запишем в виде
\begin{equation}\label{eq:q-conv}
  q_{ij}^{\mathrm{conv},\,n+1}
  =
  \sum_{\alpha\in\{g,w\}}
    \rho_{\alpha,ij}^{\,n+1}
    h_{\alpha,ij}^{\,n+1}
    \left(
      \mathbf{u}_{\alpha,ij}^{\,n+1}\cdot \mathbf{n}_{ij}
    \right)
    A_{ij}.
\end{equation}

Здесь:
\begin{itemize}
  \item \(\rho_{\alpha,ij}^{\,n+1}\)~--- интерполированная плотность фазы
        \(\alpha\) на грани \(i\text{--}j\)
        (например, среднее между ячейками или апвинд-приближение);
  \item \(h_{\alpha,ij}^{\,n+1}\)~--- удельная энтальпия фазы \(\alpha\)
        на грани;
  \item \(\mathbf{u}_{\alpha,ij}^{\,n+1}\cdot \mathbf{n}_{ij}\)~---
        нормальная к грани составляющая фильтрационной скорости фазы \(\alpha\),
        взятая на шаге \(t^{n+1}\);
  \item знак потока (вход/выход из ячейки \(V_i\)) определяется
        направлением нормали \(\mathbf{n}_{ij}\).
\end{itemize}

Таким образом, слагаемое
\[
  \nabla\cdot
  \Bigl(
    \sum_{\alpha\in\{g,w\}}
      \rho_\alpha h_\alpha \mathbf{u}_\alpha
  \Bigr)
\]
в непрерывном уравнении \eqref{eq:energy}
в дискретной форме заменяется суммой конвективных потоков
\(\sum_{j\in N(i)} q_{ij}^{\mathrm{conv},\,n+1}\).

\subsection*{Дискретное уравнение для ячейки контрольного объёма}

Подставляя выражения \eqref{eq:q-cond}, \eqref{eq:q-conv}
в уравнение \eqref{eq:energy-fvm}, получаем окончательную схему
для контрольного объёма \(V_i\):
\begin{equation}\label{eq:energy-final}
\begin{aligned}
  \frac{|V_i|}{\Delta t}
  \left(
    E_i^{n+1} - E_i^{n}
  \right)
  &+
  \sum_{j\in N(i)}
  \Biggl[
    -\lambda_{\mathrm{eff},ij}^{\,n+1}
    \left(
      \frac{T_j^{n+1} - T_i^{n+1}}{d_{ij}}
    \right)
    A_{ij}
    +
    \sum_{\alpha\in\{g,w\}}
      \rho_{\alpha,ij}^{\,n+1}
      h_{\alpha,ij}^{\,n+1}
      \left(
        \mathbf{u}_{\alpha,ij}^{\,n+1}\cdot \mathbf{n}_{ij}
      \right)
      A_{ij}
  \Biggr]
  \\
  &= - |V_i|\, L_h R_{h,i}^{n+1}.
\end{aligned}
\end{equation}

Уравнение \eqref{eq:energy-final}:
\begin{itemize}
  \item учитывает теплопроводный перенос тепла через грани
        (первое слагаемое в сумме по \(j\));
  \item учитывает конвективный перенос энтальпии фазами газа и воды
        (второе слагаемое в сумме по \(j\));
  \item учитывает тепловой эффект образования газового гидрата
        как сток энергии.
\end{itemize}

\section*{Численная схема для неразрывности}

В этом разделе приводится дискретизация уравнений сохранения массы
\eqref{eq:mass-balance} методом контрольных объёмов
и формирование общей нелинейной алгебраической системы.

\subsection*{Интегральная форма уравнения массы для контрольного объёма}

Рассмотрим контрольный объём \(V_i\) с центром в точке \(\mathbf{x}_i\).
Интегрируя локальное уравнение баланса массы фазы
\(\alpha \in \{g,w,h\}\) по \(V_i\), получаем
\begin{equation}\label{eq:mass-integral}
  \frac{\partial}{\partial t}
  \int_{V_i} \phi S_\alpha \rho_\alpha\, dV
  +
  \oint_{\partial V_i}
    \rho_\alpha \mathbf{u}_\alpha \cdot \mathbf{n}\, dA
  =
  \int_{V_i} \Gamma_\alpha\, dV ,
\end{equation}
где \(\Gamma_\alpha\)~--- объёмная плотность источника/стока массы фазы
за счёт фазового перехода.

Введём обозначение
\begin{equation}\label{eq:M-i}
  M_{\alpha,i} = \phi_i S_{\alpha,i} \rho_{\alpha,i},
\end{equation}
как массу фазы \(\alpha\) на единицу объёма пористой среды в ячейке \(V_i\).
Тогда при допущении кусочно-постоянных величин по \(V_i\) имеем
\[
  \int_{V_i} \phi S_\alpha \rho_\alpha\, dV
  \approx
  |V_i|\, M_{\alpha,i},
  \qquad
  \int_{V_i} \Gamma_\alpha\, dV
  \approx
  |V_i|\, \Gamma_{\alpha,i},
\]
где \(|V_i|\)~--- объём ячейки.

\subsection*{Дискретизация по времени (неявная схема)}

Пусть шаг по времени равен \(\Delta t = t^{n+1} - t^n\).
Используем неявный метод Эйлера (backward Euler) для производной по времени:
\[
  \frac{\partial}{\partial t}
  \int_{V_i} \phi S_\alpha \rho_\alpha\, dV
  \approx
  \frac{|V_i|}{\Delta t}\left(
    M_{\alpha,i}^{n+1} - M_{\alpha,i}^{n}
  \right),
\]
где верхний индекс \(n\) относится к моменту времени \(t^n\).

Поток через грань \(i\text{--}j\) обозначим как
\[
  F_{\alpha,ij}
  =
  \int_{A_{ij}} \rho_\alpha \mathbf{u}_\alpha\cdot\mathbf{n}_{ij}\, dA
  \approx
  \rho_{\alpha,ij}\,
  (\mathbf{u}_{\alpha,ij}\cdot\mathbf{n}_{ij})\,A_{ij},
\]
где \(A_{ij}\)~--- площадь грани,
\(\mathbf{n}_{ij}\)~--- единичная нормаль,
направленная из ячейки \(i\) в ячейку \(j\).

Тогда дискретное уравнение баланса массы в ячейке \(V_i\) имеет вид
\begin{equation}\label{eq:mass-fvm}
  \frac{|V_i|}{\Delta t}\left(
    M_{\alpha,i}^{n+1} - M_{\alpha,i}^{n}
  \right)
  +
  \sum_{j\in N(i)}
    F_{\alpha,ij}^{\,n+1}
  =
  |V_i|\,
  \Gamma_{\alpha,i}^{n+1},
\end{equation}
где \(N(i)\)~--- множество соседних ячеек, имеющих общую грань с \(V_i\).

\subsection*{Массовый поток и дискретный закон Дарси}

Фильтрационная скорость фазы \(\alpha\) задаётся законом Дарси
в форме с эффективной фазовой проницаемостью:
\[
  \mathbf{u}_\alpha
  =
  -\frac{k_{\mathrm{eff},\alpha}}{\mu_\alpha}
  \left(
    \nabla p_\alpha - \rho_\alpha \mathbf{g}
  \right),
\]
где \(k_{\mathrm{eff},\alpha}=k_{\mathrm{eff},\alpha}(S_g,S_w,S_h,t)\)
учитывает абсолютную проницаемость, относительную фазовую
проницаемость и кольматацию.

Дискретная нормальная компонента скорости на грани \(i\text{--}j\):
\begin{equation}\label{eq:darcy-disc}
  \mathbf{u}_{\alpha,ij}^{\,n+1}\cdot \mathbf{n}_{ij}
  =
  -\frac{k_{\mathrm{eff},\alpha,ij}^{\,n+1}}{\mu_{\alpha,ij}^{\,n+1}}
    \left(
      \frac{p_{\alpha,j}^{n+1}-p_{\alpha,i}^{n+1}}{d_{ij}}
      - (\rho_{\alpha,ij}^{n+1}\mathbf{g})\cdot \mathbf{n}_{ij}
    \right),
\end{equation}
где \(d_{ij}\)~--- расстояние между центрами ячеек \(V_i\) и \(V_j\),
а индекс \(ij\) означает интерполяцию величин на грань.

Массовый поток через грань, используемый в \eqref{eq:mass-fvm}:
\begin{equation}\label{eq:F-alpha-ij}
  F_{\alpha,ij}^{\,n+1}
  =
  \rho_{\alpha,ij}^{\,n+1}
  \left(
    \mathbf{u}_{\alpha,ij}^{\,n+1}\cdot \mathbf{n}_{ij}
  \right)
  A_{ij}.
\end{equation}

\subsection*{Источник массы из-за фазового перехода}

Как было введено ранее, скорость образования гидрата \(R_h\)
определяется, например, по модели Ленгмюра:
\begin{equation}\label{eq:Rh-disc}
  R_h^{n+1}
  =
  \rho_h k_L\left(
    S_h^{\mathrm{eq}}(p_g^{n+1},T^{n+1}) - S_h^{n+1}
  \right).
\end{equation}

Тогда объёмные источники масс для фаз:
\[
  \Gamma_g^{n+1} = -\nu_g R_h^{n+1},\qquad
  \Gamma_w^{n+1} = -\nu_w R_h^{n+1},\qquad
  \Gamma_h^{n+1} = +\nu_h R_h^{n+1}.
\]

\subsection*{Дискретные уравнения баланса масс для газа, воды и гидрата}

Используя \eqref{eq:mass-fvm} и \eqref{eq:M-i}, получаем
для газовой фазы:
\begin{equation}\label{eq:mass-g-disc}
  \frac{|V_i|}{\Delta t}\left(
    M_{g,i}^{n+1} - M_{g,i}^{n}
  \right)
  +
  \sum_{j\in N(i)} F_{g,ij}^{n+1}
  =
  -|V_i|\, \nu_g R_{h,i}^{n+1},
\end{equation}
для водной фазы:
\begin{equation}\label{eq:mass-w-disc}
  \frac{|V_i|}{\Delta t}\left(
    M_{w,i}^{n+1} - M_{w,i}^{n}
  \right)
  +
  \sum_{j\in N(i)} F_{w,ij}^{n+1}
  =
  -|V_i|\, \nu_w R_{h,i}^{n+1},
\end{equation}
и для гидрата:
\begin{equation}\label{eq:mass-h-disc}
  \frac{|V_i|}{\Delta t}\left(
    M_{h,i}^{n+1} - M_{h,i}^{n}
  \right)
  =
  +|V_i|\, \nu_h R_{h,i}^{n+1}.
\end{equation}

Здесь
\[
  M_{\alpha,i}^{n}
  =
  \phi_i^{n} S_{\alpha,i}^{n} \rho_{\alpha,i}^{n},
  \qquad
  M_{\alpha,i}^{n+1}
  =
  \phi_i^{n+1} S_{\alpha,i}^{n+1} \rho_{\alpha,i}^{n+1},
\]
с учётом зависимости \(\phi\) и \(\rho_\alpha\) от давления,
температуры и насыщений.

\subsection*{Матричная форма полной нелинейной системы}

Сгруппируем неизвестные для каждой ячейки \(V_i\):
\[
  \mathbf{x}_i^{n+1}
  =
  \begin{bmatrix}
    p_{g,i}^{n+1} \\
    S_{g,i}^{n+1} \\
    S_{w,i}^{n+1} \\
    S_{h,i}^{n+1} \\
    T_i^{n+1}
  \end{bmatrix},
\]
причём \(S_{h,i}^{n+1} = 1 - S_{g,i}^{n+1} - S_{w,i}^{n+1}\).

Объединяя все ячейки:
\[
  \mathbf{X}^{n+1}
  =
  \begin{bmatrix}
    \mathbf{x}_1^{n+1} \\
    \mathbf{x}_2^{n+1} \\
    \vdots \\
    \mathbf{x}_N^{n+1}
  \end{bmatrix}.
\]

Дискретные уравнения баланса масс \eqref{eq:mass-g-disc}--\eqref{eq:mass-h-disc}
и дискретное уравнение энергии в форме \eqref{eq:energy-final}
можно записать в виде общей нелинейной системы:
\begin{equation}\label{eq:matrix-system}
  \mathbf{F}(\mathbf{X}^{n+1}) = \mathbf{0},
\end{equation}
где вектор невязки \(\mathbf{F}\) включает для каждой ячейки \(V_i\)
уравнения баланса массы для газа, воды, гидрата и баланс энергии.

Решение системы \eqref{eq:matrix-system} можно получать,
например, итерационным методом Ньютона:
\[
  \mathbf{J}(\mathbf{X}^{n+1,k})\,\Delta\mathbf{X}^{k}
  =
  -\mathbf{F}(\mathbf{X}^{n+1,k}),
\]
где \(\mathbf{J}\)~--- якобиан системы по \(\mathbf{X}\),
\(k\)~--- номер итерации Ньютона,
\(\Delta\mathbf{X}^{k}\)~--- поправка.
После нахождения \(\Delta\mathbf{X}^{k}\) задаётся новое приближение
\(\mathbf{X}^{n+1,k+1} = \mathbf{X}^{n+1,k} + \Delta\mathbf{X}^{k}\)
до достижения заданного критерия сходимости.

\subsection*{Возможности предлагаемой физико-математической модели}

Таким образом, предлагаемая модель включает в себя:
\begin{itemize}
  \item массовые балансы фаз \eqref{eq:mass-balance};
  \item закон Дарси в виде \eqref{eq:darcy-keff} с
        эффективной фазовой проницаемостью \eqref{eq:keff-full};
  \item уравнения для связи пористости с насыщением гидратом \eqref{eq:phi_Sh},
        что позволяет динамически вычислять изменение проницаемости коллектора;
  \item зависимость абсолютной проницаемости от пористости и насыщения
        газовым гидратом \eqref{eq:kabs_Sh};
  \item кинетику образования гидрата по Ленгмюру \eqref{eq:hydrate-kinetics},
        эволюцию \(S_h\) \eqref{eq:Sh-evol-langmuir};
  \item баланс энергии \eqref{eq:energy} с источником-стоком
        скрытой теплоты фазового перехода \(-L_h R_h\);
  \item дополнительную динамику кольматации по Леонтьеву
        через \(F_{\mathrm{colm}}\) \eqref{eq:leontiev}, \eqref{eq:leontiev-sol}.
\end{itemize}

Данная модель позволяет:
\begin{itemize}
  \item учитывать зависимость абсолютной проницаемости \(k_{\mathrm{abs}}\)
        от насыщения порового пространства газовым гидратом;
  \item учитывать изменение относительной фазовой проницаемости \(k_{r\alpha}\);
  \item на основе \(k_{\mathrm{abs}}, F_{\mathrm{colm}}, k_{r\alpha}\)
        описывать динамику работы скважины в условиях аномально низких
        пластовых давлений и температур, прогнозировать вынужденные
        остановки для геолого-технических мероприятий, подбирать оптимальные
        режимы работы для противодействия образованию газовых гидратов
        в призабойной зоне.
\end{itemize}

\end{document}
